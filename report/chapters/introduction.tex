\chapter{Einführung}
\label{cha:Introduction}

Dies ist ein Beispiel Kapitel, welches zeigt, wie Code mit minted und Syntax highlighting funktioniert.

\section{Minted}


\begin{longlisting}
    \caption{Referenz zu echten file}
    \label{csharp:apm_calc1}
    \inputminted[firstline=7, lastline=44]{csharp}{code/program.cs}
\end{longlisting}


\begin{longlisting}
    \caption{Code direkt im .tex file}
    \label{csharp:TPLThreadLocalForEach}
\begin{minted}{csharp}
// <int> is the type of the source elements,
// <long> is the type of the partition-local variable
Parallel.ForEach<int, long>(nums, // source collection
    () => 0, // method to initialize the partition variable (subtotal)
    (j, loop, subtotal) => { 
        subtotal += j; // j is the value of the current item
        return subtotal; // value to be passed to next iteration
    },
    (subtotal) => Interlocked.Add(ref total, subtotal) 
);
\end{minted}
\end{longlisting}

\begin{longlisting}
    \caption{Andere Sprache}
    \label{kotlin:tryCatchAsyncWithScope}
\begin{minted}{kotlin}
coroutineScope {
    try {
        val result = coroutineScope {
            val a = async { 
                throw IllegalStateException()
                5 // needed to compile
            }
            val b = async { 
                delay(100)
                5 
            }

            a.await() + b.await() // return result to the scope
        }
        println(result) // will not be reached
    } catch (e:Exception) {
        println("handled: $e") // => handled: java.lang.IllegalStateException
    }
}
\end{minted}
\end{longlisting}

Aktuell wird nur Kotlin und C\# unterstützt. Um eine neue Sprache hinzuzufügen, muss in CustomListings.sty ein neues Environment zum Schluss angefügt werden.




